\documentclass[11pt, a4paper]{article}

\usepackage[french]{babel}
\usepackage{fancyhdr}
\usepackage[margin=.8in]{geometry}

\usepackage{Style/TeXingStyle}

\pagestyle{fancy}
\renewcommand{\headrulewidth}{1.5pt}
\renewcommand{\footrulewidth}{0.5pt}
\fancyhead[L]{EPITA\_ING2\_2018\_S8}
\fancyhead[R]{Majeure SCIA}
\fancyhead[C]{OCVX1}
\fancyfoot[C]{\thepage}
\fancyfoot[L]{2017}
\fancyfoot[R]{\textbf{Chargé de cours :} \textsc{Bashar~DUDIN}}

\pretitle{\vspace{-2.5\baselineskip} \begin{center}}
\title{%
  { \huge Feuille d'exercices}%
}
\posttitle{
\end{center}
\rule{\textwidth}{1.5pt}
\vspace{-3\baselineskip}
}
\author{}
\date{}

\pdfinfo{
   /Author (Bashar Dudin)
   /Title  (Exercices OCVX - 2017)
   /Subject (Optimisation convexe)
}

\begin{document}

\maketitle\thispagestyle{fancy}

\section{Calcul différentiel élémentaire}

Cette série de questions est calculatoire. Savoir calculer la
différentielle, le gradient (plus généralement la jacobienne) ou la
dérivée directionnelle d'une fonction à plusieurs variable et/ou
mulitivariées en tout point (ou du moins là où cela fait sens est
nécessaire. Cela permet d'écrire par exemple les conditions KKT d'un
problème d'optimisation.

\begin{question}
  Expliciter la jacobienne, en tout point ou cela fait sens, des
  expressions différentiables suivantes :
  \begin{enumerate}
  \item $f(x, y) = e^{xy}(x+y)$ ;
  \item $g(x, y, z) = (x + y\ln(z), xyz)$ ;
  \item $h(x, y, z) = \left(\dfrac{x^2\sin(xy)}{z^2 + 2}, \tan(xyz)\right)$.
  \end{enumerate}
\end{question}
\begin{solution}
  \begin{enumerate}
  \item $f$ est différentiable en tout point de $\R^2$ car elle est
    somme, produit et composée de fonctions différentiable sur tout
    $\R^2$. On a
    \[
    \mathrm{J}_f(x, y) = \Big(e^{xy}\big(y^2+ xy + 1\big), e^{xy}\big(x^2+ yx + 1\big)\Big).
    \]
  \item $g$ est différentiable sur son domaine de définition,
    c'est-à-dire $\R\times \R\times \R_+^*$, chacune de ses
    composantes est la somme et produit de fonctions
    différentiables. On a
    \[
    \mathrm{J}_g(x, y, z) = \begin{pmatrix} 1 & \ln(z) & y/z \\ yz & xz & yx \end{pmatrix}.
    \]
  \item $h$ est différentiable sur tout $\R^3$, car chaque composante
    est somme produit et composée de fonctions différentiables. On a
    \[
    \mathrm{J}_h(x, y, z) =
    \begin{pmatrix}
      \frac{x}{z^2 + 2}\big(2\sin(xy) + xy(\cos(xy)\big)
      & \frac{x^3}{z^2+2}\cos(xy)
      & \frac{-2zx^2}{(z^2+2)^2}\sin(xy) \\
      \big(1+\tan^2(xyz)\big)yz
      & \big(1+\tan^2(xyz)\big)xz
      & \big(1+\tan^2(xyz)\big)xy
    \end{pmatrix}
    \]
  \end{enumerate}
\end{solution}

\begin{question}
  Calculer la différentielle des fonctions suivantes:
  \begin{enumerate}
  \item $f : \R^n \rightarrow \R$ donnée par l'expressoin
    $X \mapsto \cos^2(X^TAX)$, où $A$ est une matrice carrée de taille
    $(n, n)$ ;
  \item $g : M_n(\R) \rightarrow \R$ donnée par
    $A \mapsto \mathrm{tr}(A)^2$ ;
  \item $h : M_n(\R) \rightarrow M_n(\R)$ donnée par l'expressoin
    $A \mapsto A^TA^2$.
  \end{enumerate}
\end{question}

\begin{solution}
  \begin{enumerate}
  \item En utilisant la formule de différentiation des composées, en
    tout point $X \in \R^n$ on obtient
    \begin{align}
    \Di{f}(X) & = 2\cos(X^TAX)\big(-\sin(X^TAX)\big)(2X^TA) \\
              & = -2\sin(2X^TAX)X^TA
    \end{align}
    Il est sous entendu ici que l'application linéaire $\Di{f}(X)$ en
    $h \in \R^n$ est donnée par
    \[
    \Di{f}(X)(h) = 2\sin(2X^TAX)X^TAh.
    \]
    On s'est permis d'aller plus vite sur l'écriture, en suivant la
    formule différentiation des composées de fonctions.
  \item Cette fois la variable de dérivation est une matrice (qu'on
    peut identifier à un vecteur dans $\R^{n^2}$, on n'est pas en
    train de sortir du cadre du cours). On est face à une composée de
    fonctions. En rappelant que $\mathrm{tr}$ est linéaire en $A$
    (donc se différentielle c'est elle même) on a
    \[
    \Di{g}(A) = 2\mathrm{tr}(A)\mathrm{tr}.
    \]
    En $H\in M_n(\R)$, cette différentielle prend la valeur
    \[
    \Di{g}(A)(H) = 2\mathrm{tr}(A)\mathrm{tr}(H).
    \]
  \item On va procéder à la main dans ce dernier cas. Sans raison
    particulière à part le changement.
    \begin{align}
      \big(A+H\big)^T\big(A+H\big)^2 & = \big(A+H\big)^T\big(A^2 + AH + HA + H^2) \\
                                     & = A^TA + A^TAH + A^THA + H^TA^2 + \underbrace{H^TAH + H^THA + H^TH^2}_{o(\|H\|_2)}
    \end{align}
    Par identification on trouve
    \[
    \Di{h}(A)(H) = A^TAH + A^THA + H^TA^2.
    \]
    Attention au fait que la mulitplication matricielle n'est pas commutative.
  \end{enumerate}
\end{solution}

\begin{question}
  Soit $f : \R^3 \to \R$ une fonction différentiable en tout point de
  $\R^3$. On considère la fonction $g : \R^3 \to \R$ donnée pour tout
  $(x, y, z) \in \R^3$ par
  \[
  g(x, y, z) = f(x-y, y-z, z-x).
  \]
  La fonction $g$ est différentiable en tout point car composée de
  fonction différentiables. Montrer qu'on a la relation
  \[
  \frac{\pa g}{\pa x}(a,b, c) + \frac{\pa g}{\pa y}(a, b, c) + \frac{\pa
    g}{\pa z}(a, b, c) = 0
  \]
  pour tout point $(a, b, c) \in \R^3$.
\end{question}

\begin{solution}
  Il y a deux manières d'aborder cette question :
  \begin{itemize}
  \item[\textbullet]
    utiliser les formules des dérivées partielles de composées, qu'on
    n'a pas vraiment apprises ;
  \item[\textbullet]
    calculer les jacobienne de chacun des termes de la composée en
    suivant la formule de différentiation d'une composée ;
  \end{itemize}
  C'est cette dernière approche qu'on va suivre. La fonction $g$ est
  la composée de $f$ et de la fonction
  $\phi : (x, y, z) \mapsto (x-y, y-z, z-x)$. On note $v$ le vecteur
  $v = (x-y, y-z, z-x)$. La jacobienne de $g$ en tout point
  $(x, y, z)$ est donc donnée par
  \begin{align}
    \mathrm{J}_g(x, y, z) & = J_{f}(x, y, z)\times J_\phi(x, y, z)\\
                          & =
                            \begin{pmatrix}
                              \frac{\pa f}{\pa x}v &
                              \frac{\pa f}{\pa y}v &
                              \frac{\pa f}{\pa z}v
                            \end{pmatrix}
                            \begin{pmatrix}
                              1 & 0 & -1 \\
                              -1 & 1 & 0 \\
                              0 & -1 & 1
                            \end{pmatrix}\\
                          & =
                            \begin{pmatrix}
                              \frac{\pa f}{\pa x}v
                              - \frac{\pa f}{\pa y}v &
                              \frac{\pa f}{\pa y}v
                              - \frac{\pa f}{\pa z}v &
                              \frac{\pa f}{\pa z}v
                              -\frac{\pa f}{\pa x}v
                            \end{pmatrix}\\
                          & =
                            \begin{pmatrix}
                              \frac{\pa g}{\pa x}(x, y, z) &
                              \frac{\pa g}{\pa y}(x, y, z) &
                              \frac{\pa g}{\pa z}(x, y, z)
                            \end{pmatrix}
  \end{align}
  La relation qu'on attend correspond au fait que la somme des
  colonnes de la jacobienne de $g$ en $(a, b, c)$ soit nulle. Ce qui
  est clair de ce qu'on vient mettre en évidence.
\end{solution}

\begin{question}
  On considère la fonction définie sur $\R^2$ par
  $f(x, y) = \frac{x^2\cos(y)}{y + \sin^2(x) + 1}$. Pour tout
  $\vartheta \in \R$ on désigne par $w_\vartheta$ le vecteur
  $(\vartheta, \vartheta^2)$.
  \begin{enumerate}
  \item Calculer la dérivée directionnelle $\pa_{w_\vartheta}f(0, 0)$
    en fonction de $\vartheta$ ;
  \item Étudier cette fonction de $\vartheta$.
  \item Pouvez-vous interpréter vos résultats géométriquement?
  \end{enumerate}
\end{question}
\begin{solution}
  \begin{enumerate}
  \item
    On cherche dont la dérivée en $0$ de la fonction numérique
    \[
    t \xrightarrow{f_{W_\theta}} f\big((0,0) + tw_\theta\big)
    \]
    On a
    \[
    f_{w_\theta}(t) = \dfrac{t^2\theta^2\cos(t\theta^2)}{t\theta^2 + \sin^2(t\theta) + 1}
    \]
    C'est une fonction dérivable sur $\R$. En y réfléchissant un peu, et
    en suivant la règle de dérivation d'un produit (ou d'un quotient,
    comme vous voulez) vous pouvez remarquer que le dénominateur de la
    dérivée ne s'annule pas et qu'on a, par contre, un $t$ en facteur au
    numérateur. La dérivée en $0$ est donc nulle.
  \item
    Pour l'étude de fonction, c'est un fail. Je ne pensais pas que
    j'aurais quelque chose d'aussi simple en bidouillant mes
    coefficients. C'est \c{c}a de devenir vieux, on ne réfléchit plus.
  \item
    Cela signifie que la fonction $f$ a des dérivées directionnelles
    constantes dans toutes les directions définies par l'origine et
    les points de la parabole $y = x^2$.
  \end{enumerate}
\end{solution}

\section{Géométrie différentielle élémentaire}

Dans la suite, on travaille quelques notions de géométrie
différentielle élémentaire.

\begin{question}
  Trouver les points sur le paraboloïde $z = 4x^2 + y^2$ où le plan
  tangent est parallèle au plan $x + 2y + z = 6$. Faire de même avec le plan
  $3x + 5y -2z = 5$.
\end{question}
\begin{solution}
  Le plan tangent en un point $(x, y, z)$ est l'orthogonal du gradient
  de $f : (x, y, z) \mapsto z - 4x^2 - y^2$ en ce point. Pour que le
  plan tangent soit parallèle à tout autre plan, il faut et il suffit
  que le gradient soit colinéaire à un vecteur normal de
  celui-ci. Ainsi, dans le premier cas
  \begin{align}
    (1, 2, 1) = \lambda(-8x, -2y, 1) & \Longleftrightarrow x = -\frac{1}{8}, \quad y = -1
  \end{align}
  Comme $x$, $y$, $z$ doit satisfaire l'équation de départ (le point
  appartient au paraboloïde de départ), on trouve
  \[
  x = -\frac{1}{8}, \quad y = -1, \quad z = \frac{-15}{16}.
  \]
  Dans le second cas on trouve :
  \[
  x = \frac{3}{4}, \quad y = 5, \quad z = 27\frac{1}{4}.
  \]
\end{solution}

\begin{question}
  Un étudiant malheureux trouve pour plan tangent à la surface donnée
  par $z = x^4-y^2$ au point $(2, 3, 7)$ la réponse
  \[
  z = 4x^3(x-2) - 2y(y-3) + 7.
  \]
  \begin{enumerate}
  \item Sans calcul, pourquoi est-ce faux?
  \item Donner la réponse correcte.
  \item D'où venait la confusion de l'étudiant?
  \end{enumerate}
\end{question}

\section{Programmes linéaires en petite dimension}

Les programmes linéaires en dimension $2$ et parfois $3$ peuvent être
résolus par une démarche géométrique qui suit les grandes lignes de
l'optimisation convexe : on initialise en un point, on y calcule le
gradient puis on met à jour l'estimation à l'aide de cette
information.

\begin{question}
  Donner un exemple d'un programme linéaire:
  \begin{enumerate}
  \item non borné
  \item de lieu admissible non-borné mais de solution fini
  \item ayant une infinité de solution
  \item ayant une unique solution.
  \end{enumerate}
\end{question}

\noindent On considère le programme linéaire (\emph{PL}) suivant :
\begin{displaymath}
  \begin{linearProg} {
      minimiser
    }{
      $f_0(x_1, x_2) = 3x_1 + 2x_2$
    }{
      \systeme{
        x_1 - x_2 \leq 0,
        4x_1 - x_2 \leq 1,
        -x_1 - x_2 \leq -5
      }
    }
  \end{linearProg}
\end{displaymath}
\begin{question}
  \begin{enumerate}
  \item Représenter le lieu admissible de (\emph{PL}) dans le plan
    euclidien.
  \item
    \begin{enumerate}
    \item[a.]  Tracer la courbe de niveau $6$ de la fonction objectif
      de (\emph{PL}). Elle sera notée $C_6$.
    \item[b.]  Indiquer les demi-espaces positif et négatif défini par
      $C_6$. Dans quelle direction doit-on translater $C_6$ afin de
      minimiser $f_0$.
    \end{enumerate}
  \item
    Tracer la courbe de niveau qui réalise le minimum de (\emph{PL})
    et calculer l'unique point optimal de (\emph{PL}). Quelle est la
    valeur optimale de (\emph{PL})?
  \item Résumer votre démarche en rappelant la condition d'optimalité
    d'une solution de (\emph{PL}).
  \end{enumerate}
\end{question}

On considère désormais le programme linéaire (\emph{PL'}) donné par
\[
  \begin{linearProg} {
      minimiser
    }{
      $f_0(x_1, x_2) = x_1 + 2x_2$
    }{
      \systeme{
        x_2 \leq 1,
        x_1 \leq 1,
        -x_2 \leq 1,
        -x_1 \leq 1
      }
    }
  \end{linearProg}
\]
\begin{question}
  Résoudre (\emph{PL'}) en suivant la démarche précédente.
\end{question}

\section{Problèmes d'optimisation}

\subsection{Baby examples}

L'approche géométrique dans le cas des programmes linéaires de petites
dimensions s'étend à certains problèmes d'optimisations simples.

\begin{question}
On considère la fonction différentiable $f : \R^2 \to \R$ donnée par
\[
f(x, y) = 3x^2 + y^2.
\]
\begin{enumerate}
\item Représenter les courbes de niveaux $2$ et $4$ de $f$ dans le
  plan euclidien.
\item À quel lieu correspond la condition $f(x, y) \leq 4$ ?
\item On s'intéresse au problème d'optimisation (\emph{P1})
  \[
  \begin{PbOptim}{
      minimiser
    }{
      $2x + y$
    }{
      $3x^2 + y^2 \leq 4$
    }
  \end{PbOptim}
  \]
  Représenter la courbe de niveau de la fonction objective qui
  correspond à la valeur optimale de (\emph{P1}).
\item Comment trouver le point optimale correspondant à (\emph{P1}) ?
  Faites le calcul.
\item Expliciter le problème dual (\emph{P1D}).
\item Résoudre (\emph{P1D}) et comparer sa solution avec celle de
  (\emph{P1}).
\end{enumerate}
\end{question}

\noindent On s'intéresse désormais à la résolution du problème
d'optimisation (\emph{P2})
  \[
  \begin{PbOptim}{
      minimiser
    }{
      $x^2 + xy + 2y^2$
    }{
      $3x^2 + y^2 \leq 4$
    }
  \end{PbOptim}
  \]
  \begin{question}
    \begin{enumerate}
    \item Écrire le problème dual (\emph{P2D}) de (\emph{P2}). Lequel
      est le plus simple à résoudre?
    \item Résoudre le (\emph{P2}).
    \item En écrivant les conditions KKT pour (\emph{P2}) préciser le
      lien entre des solutions du dual et du primal.
    \end{enumerate}
\end{question}

\subsection{Conditions KKT}

Dans cette partie vous avez le droit d'utiliser tous les éléments du
cours à votre disposition. Les conditions KKT sont certainement ce que
vous avez vu de plus complet. N'oubliez pas que les conditions KKT ne
donnent qu'une condition nécessaire dans le cas d'un problème
d'optimisation non convexe.

\begin{question}
  Résoudre le problème d'optimisation
  \[
  \begin{linearProg}{
      minimiser
    }{
      $x^2 -14x + y^2 - 6y - 7$
    }{
      \systeme{
        x + y \leq 2,
        x + 2y \leq 3
      }
    }
  \end{linearProg}
  \]
\end{question}

\noindent On s'intéresse à un problème plus géométrique!

\begin{question}
  \begin{enumerate}
  \item
    Déterminer le rectangle de plus grande surface inscrit dans
    l'ellipse d'équation
    \[
    \frac{x^2}{a} + \frac{y^2}{b} = 1.
    \]
  \item
    Faire de même avec le parallélépipède de plus grand volume inscrit
    dans l'ellipsoïde d'équation
    \[
    \frac{x^2}{a} + \frac{y^2}{b} + \frac{z^2}{c} = 1.
    \]
  \end{enumerate}
\end{question}

\noindent Question d'économie domestique.

\begin{question}
  Une ressource disponible en quantité d doit être affectée à trois
  activités en quantités $x_1$ , $x_2$ et $x_3$
  respectivement. L’allocation de $x_k$ unités de ressource à
  l’activité $k$ procure une recette évaluée par
  $f_k(x_k) = 8x_k - k{x_k}^2$. Calculer l'allocation qui procure la
  plus grande recette dans les deux éventualités suivantes:
  \begin{enumerate}
  \item la quantité $d$ est complètement utilisée ;
  \item l'excédant quand $d$ n'est pas épuisé est revendu au prix $p$.
  \end{enumerate}
  A-t-on intérêt à utiliser toutes les ressources disponibles?
\end{question}

\section{Un peu de généralité et de biblio}

\begin{question}
  Qu'est-ce que la méthode LASSO? Comment exprimer celle-ci comme un
  programme d'optimisation? Comment aborder la résolution de ce
  dernier?
\end{question}

\pretitle{\vspace{-2\baselineskip} \begin{center}}
\title{%
  { \huge Solutions des exercices}%
}
\posttitle{
\end{center}
  \vspace{.5\baselineskip}
  \rule{\textwidth}{1.5pt}
  \vspace{-5\baselineskip}
}

\maketitle\thispagestyle{fancy}

\noindent Vous trouverez dans la suite solutions et indications d'une
partie des exercices de la feuille. Ceci étant majoritairement
accessibles il vous est suffisant de comparer votre travail au
résultats que vous retrouverez dans la suite.

\printsolutions

\end{document}

%%% Local Variables:
%%% mode: latex
%%% TeX-master: t
%%% End:
